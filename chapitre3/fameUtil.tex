\newpage
\section{Frameworks utilisés}

\subsection{Angular}
\begin{figure}[h!]  
 \centering
    \includegraphics[width=0.2\textwidth]{chapitre3/Figures/angularLogo.png}
\end{figure}
Angular est un framework JavaScript conçu pour créer des applications web et mobiles. Il est considéré comme un langage "côté client", ce qui permet de gérer l’interface utilisateur de chaque page de façon dynamique et vient en complément aux langages côté serveur.\\
Parmi ces caractéristiques, on trouve : 
\begin{itemize}[label=\textbullet]
\item L'utilisation des nouveaux standards du web
\item La modularité; les applications sont sous forme de blocks réutilisables
\item L'intègration par défaut du support de communication avec les services back-end
\end{itemize}


\subsection{Spring Boot}
\begin{figure}[h!]  
 \centering
    \includegraphics[width=0.2\textwidth]{chapitre3/Figures/boot.png}
\end{figure}
Spring Boot est un sous projet de Spring qui vise à rendre Spring plus facile d'utilisation en élimant plusieurs étapes de configuration.\\ Spring Boot apporte à Spring une très grande simplicité d'utilisation :
\begin{itemize}[label=\textbullet]
\item Il facilite notamment la création, la configuration et le déploiement d'une application complète.
\item Il permet de déployer très facilement une application dans plusieurs environnements sans avoir à écrire des scripts. 
\item Il possède un serveur d'application Tomcat embarqué afin de faciliter le déploiement d'une application web.
\end{itemize}
\newpage

\subsection{Spring Data}
\begin{figure}[h!]  
 \centering
    \includegraphics[width=0.3\textwidth]{chapitre3/Figures/data.png}
\end{figure}
Spring Data est un framework qui permet de fournir une implémentation de la couche d’accès aux données pour une application Spring. \\
Il vise à faciliter l’utilisation des technologies d’accès aux données, des bases de données relationnelles et non relationnelles. Spring Data permet de ne pas réinventer la roue de l’accès aux données à chaque nouvelle application et donc de se concentrer sur la partie métier.


\subsection{JUnit}
\begin{figure}[h!]  
 \centering
    \includegraphics[width=0.2\textwidth]{chapitre3/Figures/junit.png}
\end{figure}
JUnit est un framework de développement et d'exécution de tests unitaires pour le langage de programmation Java. Son principal intérêt est de s'assurer que le code répond toujours aux besoins même après d'éventuelles modifications. 
JUnit est intégré par défaut dans les environnements de développement intégré Java tels qu’Eclipse.