\section{Le traitement des graphes distribuées}
L'analyse des graphes est un ensemble de techniques analytiques qui permettent d'explorer les relations entre des entités qui nous intéressent, telles que des organisations, des personnes et des transactions [b.9]. Les entrepôts de données de graphes peuvent efficacement modéliser, explorer et extraire des données avec des interrelations complexes. L'analyse des graphes et prometteuse, en raison de la nécessité de poser des questions complexes à travers des données complexes, ce qui n'est pas toujours pratique ni même possible à l'échelle en utilisant des requêtes SQL.\\
Il y a trois méthodes principales pour approcher le traitement des graphes :

\begin{itemize}[label=\textbullet]
\item  Computational graph analytics : Itérer le graphe plusieurs fois et calculer certaines propriétés mathématiques. L’exemple le plus connue pour utiliser cette approche et l’algorithme PageRank de Google, qui permet d’ordonner les pages web selon priorité. PageRank modélise le web comme un graphe ou les pages sont les sommets et les liens URL entre les pages sont les arrêts.
\item  Graph querying and pattern matching : Exécuter des requetés sur le graphe pour trouver des sous-graphes qui correspondent au pattern spécifié par l’utilisateur. Cette approche est similaire aux requêtes SQL sur base des données relationnels.
\item  Graph ML : Il consiste à utiliser les algorithmes du machine learning pour trouver des informations structurelles latente dans les graphes, comme les clusters des personnes avec des intérêts similaires dans un graph sociale.
\end{itemize}

PGX support premièrement les deux premières approches, la première puisque il offre plus que 40 algorithmes implémentés et laisse la possibilité aux utilisateurs aussi d’implémenter leur propre algorithmes via le Domain Spécific Language(DSL) GreenMarl.
