\chapter{REST API}
\begin{figure}[h!]  
 \centering
    \includegraphics[width=0.6\textwidth]{annexe/Figures/rest.png}
\end{figure}
REST, acronyme de Representational State Transfer, est un style d’architecture logiciel qui définit la communication entre les différentes parties du web. L’échange est basé sur des requêtes client et serveur. Un client interroge le serveur par une requête HTTP, et le serveur renvoie une réponse. Cette interrogation se fait suivant différentes méthodes : 
\begin{itemize}[label=\textbullet]
\item GET pour la lecture des données
\item POST pour les créer
\item PUT pour les mettre à jour 
\item DELETE pour les supprimer
\end{itemize}

REST n’impose pas un format d’échange entre le client et le serveur. La réponse envoyée par le serveur peut avoir plusieurs présentations : HTML, XML, JSON, .... C’est au développeur de définir quel format pour la représentation de réponse le convient le mieux