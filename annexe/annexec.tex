
    \chapter{Outils de développement}
\section{Astah}
\begin{figure}[h!]  
 \centering
    \includegraphics[width=0.15\textwidth]{annexe/Figures/astah.png}
\end{figure}
Astah est un outil de modélisation UML (Unified Modeling Language). ). Il contient toutes les fonctions nécessaires à la conception et au développement de systèmes logiciels orientés objets ainsi qu'à leur documentation et gestion.

\section{Eclipse}
\begin{figure}[h!]  
 \centering
    \includegraphics[width=0.25\textwidth]{annexe/Figures/eclipse.png}
\end{figure}
Eclipse est un environnement de développement intégré (IDE) s'appuyant principalement sur Java. Il propose un certain nombre de raccourcis et d'aide pour simplifier la programmation. Eclipse est gratuit et disponible pour la plupart des systèmes d'exploitation.

\newpage

\section{Git}
\begin{figure}[h!]  
 \centering
    \includegraphics[width=0.2\textwidth]{annexe/Figures/git.png}
\end{figure}
Git est un logiciel qui permet le stockage d'un ensemble de fichiers en conservant la chronologie de toutes les modifications effectuées. C'est un outil permettant à chacun de travailler à son rythme, de façon désynchronisée des autres, puis d'offrir un moyen de s'échanger leur travaux respectifs.

\section{MySQL}
\begin{figure}[h!]  
 \centering
    \includegraphics[width=0.2\textwidth]{annexe/Figures/mysql.png}
\end{figure}
MySQL est un système de gestion de base de données relationnelle (SGBDR), qui utilise le langage SQL. C'est un des SGBDR les plus utilisés.  Il est un serveur multi-thread (exécution de plusieurs processus en même temps) et multi-utilisateur qui fonctionne aussi bien sur Windows que sur Linux ou Mac OS.

\section{Postman}
\begin{figure}[h!]  
 \centering
    \includegraphics[width=0.2\textwidth]{annexe/Figures/postman.png}
\end{figure}
Postman est un outil puissant pour les API de prototypage. Il présente une interface graphique conviviale pour construire des requêtes et lire des réponses.  Postman facilite le test, le développement et la documentation des API, permettant aux utilisateurs de composer rapidement des requêtes HTTP simples ou complexes.

\newpage

\section{WebStorm}
\begin{figure}[h!]  
 \centering
    \includegraphics[width=0.15\textwidth]{annexe/Figures/webstorm.png}
\end{figure}
WebStorm est un environnement de développement intégré (IDE) pour les langages Web (HTML, CSS et JavaScript). Il offre diverses fonctionnalités comme :
\begin{itemize}[label=\textbullet]
\item Inspection du code et correction rapidet
\item Navigation rapide du code
\item Recherche d'usage d'un code
\item Refactorisation du code 
\item Détecteur de code dupliqué 
\end{itemize}
