\chapter*{Conclusion générale et perspectives}
\addcontentsline{toc}{chapter}{Conclusion Générale}
Notre projet de fin d’études effectué au sein de HPS consiste à concevoir et réaliser une application d’automatisation des tableaux de bord et indicateurs. Cette solution a pour but de faciliter, standardiser et optimiser la génération des tableaux de bord pour l’ensemble des parties prenantes.\\ \\
Afin de réaliser ces objectifs, il s’est avéré primordial de recenser les besoins fonctionnels et techniques de l’application. Cette étape nous a permis de passer à la conception de la solution en se basant sur le formalisme UML. Et enfin la bonne gestion de la phase de réalisation, en respectant les principes de la méthodologie agile SCRUM, nous ont permis d’implémenter, tester et valider les différentes évolutions réalisées conformément aux spécifications et respectant les contraintes projet défini dans le plan qualité projet.\\ \\
En guise de perspectives : 
\begin{itemize}[label=\textbullet]
\item On envisage d'offrir à l'utilisateur la possibilité d'extraire les différentes informations sous forme excel afin de les exploiter
\item Interfaçage entre project server et l'application afin de récuperer les données en temps réel sans attendre la fin de chaque mois.
\end{itemize}