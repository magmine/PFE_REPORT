\pdfbookmark[0]{Résumé}{resume}


\chapter*{Résumé}
De nos jours, la plupart des données sont organisées sous forme de tableaux. Par exemple, si l'on considère les applications bancaires, la manière typique d'organiser la plupart des données est celle des tableaux où les clients ont des comptes, les comptes ont des transactions, et les transactions se font entre les comptes (un chiffre ici), ce type d'organisation a son application là où elle excelle et c'est pourquoi les bases de données relationnelles existent depuis les années 80. Cependant, une autre façon de voir les choses est que les clients font des transactions avec d'autres clients via leurs comptes et que les comptes sont reliés entre eux par des transactions, ce qui donne un graphe, et cette façon de voir permet aux gens de poser des questions plus intéressantes qui résident dans ces connexions, prenez par exemple une carte mentale, un des outils utiles utilisés lorsque l'on est bloqué ou que l'on a besoin d'expliquer une idée de façon intuitive. 
\\ \\
De nombreux outils ont été créés pour tirer le meilleur parti de ces nouvelles façons de représenter les données, partant des bases de données de graphes comme Neo4j, jusqu'aux outils de traitement de graphes comme PGX.D, celui qui sera abordé dans ce rapport, qui est parmi les engins de traitement de graphes les plus avancés dans ce domaine. PGX.D brille particulièrement lorsqu'il s'agit de données qui ne peuvent pas être stockées dans la mémoire d'une seule machine, ou de données dont la taille peut augmenter considérablement avec le temps, comme les graphes sociaux de Facebook et Twitter par exemple. Dans ce rapport, je présenterai ma contribution à l'effort continu d'extension des capacités de PGX.D, allant du filtrage des graphes à l'analyse des performances avec TCP-H et enfin à la prise en charge des valeurs nulles dans les "frames".
\\ \\

\noindent\textbf{Mots clés :} Filtrage des graphes, TPC-H benchmarking, NULL values

\newpage
\chapter*{Abstract}
Nowadays most data is organized in the form of tables, for instance if we look at banking application the typical way of organizing most data is tables where customers have accounts, accounts have transactions, and transactions are between accounts (a figure here), this type of organizations have their application where they excel at and that’s why relational databases exist since the 80’s. However, a different way to looking at it, is that customers are transacting with other customers via their accounts and the accounts are connected via transactions, which results in a graph, and looking at this way allows people to ask more interesting questions that reside into those relationships, take as an example a mind map one of the useful tools used when one is stuck or need to explain an idea intuitively. 
\\ \\
A lot of tools have been created to take the most from these new ways of representing data from graph databases, to graph processing tools like PGX.D, the one we will discuss in this report, which is among the most advanced graph in this field. PGX.D shines especially when it comes to data that cannot fit into the memory of a single machine, or data that can grow significantly in size by time like, Facebook, and Twitter social graphs for instance. In this report I will present my contribution to the ongoing effort of extending the capabilities of PGX.D, from graph filtering to benchmarking and finally null value support in frames
\\ \\ \\ \\
\noindent\textbf{Keywords :} Graph filtering, TPC-H benchmarking, NULL values

\newpage




