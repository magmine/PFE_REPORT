\chapter*{Introduction Générale}
\addcontentsline{toc}{chapter}{Introduction Générale}
Les banques et les institutions financières ajoutent périodiquement une variété de terminaux de point de vente, pour répondre aux demandes constantes des commerçants, il y a une croissance des volumes de terminaux de point de vente. Le besoin de l'heure est d'avoir un utilitaire de gestion central pour configurer, gérer et entretenir à distance le réseau de TPE en constante évolution. \\

C'est dans ce contexte que s'inscrit notre projet de fin d’études qui a pour objectif de mettre en place une solution qui permet en temps réel via une interface WEB (Intranet) des informations relatives à l’état de fonctionnement de tous les TPE et la mise à disposition de statistiques relatifs  à l’activité de chaque terminal.\\

Le présent rapport décrit l’essentiel du travail réalisé lors de ce projet. Il comporte cinq chapitres. Le premier définit le contexte général du projet, à savoir une présentation de l’organisme d’accueil ainsi qu’une introduction au cadre général du projet et son planning. Le deuxième chapitre décrit l’étude fonctionnelle basée sur l’ensemble des diagrammes comportementaux. Le troisième chapitre présente une étude technique détaillée du projet. Quant au quatrième, il traite l'étude conceptuelle. Le dernier chapitre est réservé à la description de mise en œuvre du projet.\\ 

Des annexes seront proposées en fin du rapport pour développer quelques aspects clés qui n’ont pas été approfondis dans les différentes parties.